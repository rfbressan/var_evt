%%%%%%%%%%%%%%%%%%%%%%%%%%%%%%%%%%%%%%%%%%%%%%%%%%%%%%%%%%%%%%%%%%%%%%%%%%%%%%%%%%%%%%%%%%%%%%
% Template Beamer 
% Based on MIT Beamer Template e Senac
% Cores verde e vermelho tentam seguir o padrão visual da Udesc
%%%%%%%%%%%%%%%%%%%%%%%%%%%%%%%%%%%%%%%%%%%%%%%%%%%%%%%%%%%%%%%%%%%%%%%%%%%%%%%%%%%%%%%%%%%%%% 

\usepackage{graphicx,url}
\usepackage{tikz}
\usepackage[brazil]{babel}   
\usepackage{textpos}
\usepackage{transparent}

\batchmode
% \usepackage{pgfpages}
% \pgfpagesuselayout{4 on 1}[letterpaper,landscape,border shrink=5mm]
\usepackage{amsmath,amssymb,enumerate,epsfig,bbm,calc,color,ifthen,capt-of}

%-------------------------Declara figura do Logo-----------------------------------------------------
\pgfdeclareimage[height=0.8cm]{Marca_Udesc}{Marca_Udesc.pdf}
%\logo{\pgfuseimage{Marca_Udesc}\vspace*{7.0cm}}


%% ------------------ Figura de fundo do slide de titulo
%\setbeamercolor{titlelike}{parent=structure}
%\makeatletter
%\setbeamertemplate{title page}
%{
%	\begin{tikzpicture}[remember picture, overlay]
%	\node[at=(current page.center)]{
%		\includegraphics[width=\paperwidth, height=\paperheight]{risk1.jpg}	
%	};
%	\end{tikzpicture}
%	%	\vbox{}
%	%	\vfill
%	%	\begingroup
%	%	\centering
%	%	\pgfsetfillopacity{0.1}
%	%	\setbeamercolor{title}{bg=black}
%	%	\begin{beamercolorbox}[sep=8pt,center]{title}	
%	%		\usebeamerfont{title}\inserttitle\par%
%	%		\ifx\insertsubtitle\@empty%
%	%		\else%
%	%		\vskip0.25em%
%	%		{\usebeamerfont{subtitle}\usebeamercolor[fg]{subtitle}\insertsubtitle\par}%
%	%		\fi%
%	%	\end{beamercolorbox}%
%	\begin{textblock*}{0.8\textwidth}(2.5em, -2.5em)
%		\centering
%		\setbeamerfont{title}{size=\Huge}
%		{\usebeamerfont{title}\usebeamercolor[bg]{title}\inserttitle\par}%
%	\end{textblock*}
%	%	\vskip3em\par
%	%	\begin{beamercolorbox}[sep=8pt,center]{author}
%	%		\usebeamerfont{author}\insertauthor
%	%	\end{beamercolorbox}
%	%	\begin{beamercolorbox}[sep=8pt,center]{institute}
%	%		\usebeamerfont{institute}\insertinstitute
%	%	\end{beamercolorbox}
%	%	\begin{beamercolorbox}[sep=8pt,center]{date}
%	%		\usebeamerfont{date}\insertdate
%	%	\end{beamercolorbox}\vskip0.5em
%	%	\endgroup
%	\vfill
%}
%\makeatother

%Global Background must be put in preamble
%\usebackgroundtemplate%
%{%
%	\transparent{0.2}\includegraphics[width=\paperwidth,height=\paperheight]{risk-management.jpg}%
%}


%-------------------------Este código faz o menuzinho bacana na parte superior do slide------------
\AtBeginSection[]
{
%  \begin{frame}<beamer>
%    \frametitle{Sumário}
%    \tableofcontents[currentsection]
%  \end{frame}
}

\AtBeginSubsection{}

% Para ativar os tópicos de forma incremental
%\beamerdefaultoverlayspecification{<+->}
% -----------------------------------------------------------------------------
% -----Página de título sem o logo -----------------------------------
%\renewcommand{\titlepage}{\setbeamertemplate{logo}{}\titlepage}
%\setbeamertemplate{logo}{}\titlepage

% ------------ Logo na parte superior direita --------------------------------
\addtobeamertemplate{frametitle}{}{%
\begin{textblock*}{100mm}(.85\textwidth,-0.9cm)
\pgfuseimage{Marca_Udesc}
\end{textblock*}}
%
% - Para criar duas colunas em algum slide da apresentacao 
% - a partir do Rmarkdown
\def\begincols{\begin{columns}}
\def\begincol{\begin{column}}
\def\endcol{\end{column}}
\def\endcols{\end{columns}}

%%---Gerador de Sumário---------------------------------------------------------
%\section[]{}
%\begin{frame}{Sumário}
%  \tableofcontents
%\end{frame}
%---Fim do Sumário------------------------------------------------------------
